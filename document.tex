\documentclass[12pt,a4paper,oneside,article]{memoir}



% LAYOUT
\usepackage{changepage}
\setulmarginsandblock{0.7\uppermargin}{0.8\lowermargin}{*}

% MATHS
\usepackage{amsmath,amsfonts,amssymb,amsbsy,commath,mathtools,calc}
\mathtoolsset{showonlyrefs,showmanualtags}

\usepackage{subfig}
% FONTS & LANGUAGE
\usepackage[usenames,dvipsnames]{color}
\definecolor{light-gray}{gray}{0.8}
\usepackage{fontspec,xltxtra,polyglossia}
\setmainlanguage{english}
\usepackage[normalem]{ulem} % have underlinings work
%\defaultfontfeatures{Ligatures=TeX}
\defaultfontfeatures{Mapping=tex-text}

\setmainfont[Ligatures={Common}, Numbers={OldStyle}]{Linux Libertine O}
%\setmainfont{Droid Sans}
%\setsansfont[Scale=MatchLowercase]{Inconsolata}
\setmonofont[Scale=0.8]{DejaVu Sans Mono}

% PDF SETUP
\usepackage[unicode,bookmarks, colorlinks, breaklinks,
pdftitle={T-61.3040: Ex 9},
pdfauthor={Ville Väänänen},
pdfproducer={xetex}
]{hyperref}
\hypersetup{linkcolor=black,citecolor=black,filecolor=black,urlcolor=MidnightBlue} 

\usepackage[backref=true, backend=biber]{biblatex}
\addbibresource{bibliography.bib}

% TABLES
\usepackage{booktabs}
\usepackage{topcapt} 
\usepackage{rccol}
\usepackage{tabularx} % requires array
\newcommand{\otoprule}{\midrule[\heavyrulewidth]}
\newcolumntype{d}[2]{R[.][.]{#1}{#2}}

\usepackage{titlesec}
\usepackage{todonotes}

%%%%%%%% OMAT KOMENNOT %%%%%%%%%%%%

\usepackage{mymath}
\usepackage{mylayout}


% PARAGRAPHS
%\usepackage{parskip}

% kuvat

\usepackage{listings}
\usepackage{mcode}
\lstset{ %
	%language=Matlab,                % choose the language of the code
	basicstyle=\footnotesize\ttfamily,% the size of the fonts that are used for the code 
	numbers=none,                   % where to put the line-numbers
	numberstyle=\footnotesize\ttfamily,      % the size of the fonts that are usedfor the line-numbers 
	stepnumber=5,                   % the step between two line-numbers. If it's 1 each line 
	aboveskip=2\medskipamount,
	belowskip=2\medskipamount,                                % will be numbered
	numbersep=-5pt,                  % how far the line-numbers are from the code
	backgroundcolor=\color{white},  % choose the background color. You must add \usepackage{color}
	showspaces=false,               % show spaces adding particular underscores
	showstringspaces=false,         % underline spaces within strings
	showtabs=false,                 % show tabs within strings adding particular underscores
	frame=l,
	framesep=0pt,
	framexleftmargin=2mm,
	rulecolor=\color{light-gray},	                % adds a frame around the code
	tabsize=2,	                % sets default tabsize to 2 spaces
	caption=,
	captionpos=t,                   % sets the caption-position to bottom
	breaklines=true,                % sets automatic line breaking
	breakatwhitespace=false,        % sets if automatic breaks should only happen at whitespace
	emptylines=*1,
	%title=\lstname,                 % show the filename of files included with
	                                % also try caption instead of title
	escapeinside={\%*}{*)},         % if you want to add a comment within your code
            % if you want to add more keywords to the set
}
 
\newcommand{\course}{S-114.4202}
\newcommand{\coursename}{Special Course in Computational Engineering II}
\newcommand{\duedate}{\today}
\newcommand{\studentid}{63527M}
\renewcommand{\title}{Lansing woods}
\author{Ville Väänänen}

\setsecnumdepth{subsubsection}
\counterwithout{section}{chapter}
\pagestyle{plain}
\makeevenhead{headings}{\course}{\Large\title}{\author / \studentid}
\makeoddhead{headings}{\course}{\Large\title}{\author / \studentid}
\makeheadrule{headings}{\textwidth}{\normalrulethickness}
\makeheadposition{headings}{flushleft}{flushleft}{flushleft}{flushleft}
\checkandfixthelayout
%\renewcommand{\thesubsubsection}{\thesubsection{\large\scshape\alph{subsubsection}}}
\newfontfamily\subsubsectionfont[Letters=SmallCaps]{Linux Libertine O}
%\titleformat{\subsection}{\large\scshape}{\alph{subsection} )}{10pt}{}
%\titleformat{\section}{\Huge}{Round \thesection}{10pt}{}
%\titleformat{\section}{\Large}{Exercise \thesection}{10pt}{}

\everymath{\displaystyle}
\begin{document}
\begin{titlingpage}
	\begin{center}
	\begin{minipage}{\textwidth}
	\begin{flushright} \large
	Ville \textsc{Väänänen}\\
	\studentid
	\end{flushright}
	\end{minipage}
	
	\vspace{8.0cm}
	\textsc{\LARGE \title}
	\HRule \\[0.19cm]
	{\large \course\: \coursename}
	
	
	\vfill
	\today
	\end{center}
\end{titlingpage}
\clearpage

\section{Data description}
It's an important question in forest ecology wether
certain tree species are spatially associated with each other
and how they respond to competition.
The Lansing Woods dataset \cite{lansing} contains the location and 
botanical classification of $2251$ trees. The data
was collected in Lansing Woods, Clinton County, Michigan USA by D.J.
Gerrard in 1969 from a square area of $282\times282$ metres.

The dataset is available in the \emph{R} package \emph{spatstat} \cite{R,spatstat}.
It's a categorically marked dataset, where the mark can have one of the values

\begin{itemize}
  \item blackoak
  	\begin{itemize}
  	  \item \emph{Quercus velutina}
  	  \item known associates: whiteoak, redoak, hickory, maple
  	\end{itemize} 
  \item redoak
   \begin{itemize}
  	  \item \emph{Quercus rubra}
  	  \item known associates: whiteoak, blackoak
  	\end{itemize}  
  \item whiteoak
   \begin{itemize}
  	  \item \emph{Quercus alba}
  	  \item known associates: whiteoak, redoak
  	\end{itemize}    
  \item hickory
   \begin{itemize}
  	  \item \emph{Carya}
  	  \item known associates: \todo[fancyline]{Look it up} 
  	\end{itemize}  
  \item maple
   \begin{itemize}
  	  \item \emph{Acer}
  	  \item known associates: \todo[fancyline]{Look it up}
  	\end{itemize}  
  \item misc \todo[fancyline]{find the original article}
\end{itemize}  

The interesting questions will be:
\begin{itemize}
  \item do the associations known \emph{a priori} show in the data
  \item do some species avoid some other species
  \item clustering behavior inside and between the species
\end{itemize}
The dataset is plotted in figure~\ref{fig:lansing_separate}. 


\placefig[p]{lansing_separate}{0.95}{The Lansing woods dataset plotted
by separating the data by the marks (species)}
\placefig[p]{lansing_intensity_separate}{0.95}{Independent smoothed intensity estimates}
\placefig[p]{lansing_intensity_relative}{0.95}{Smoothed intensity estimates
across all species}

\section{Intensity analysis}
It's obvious just by looking at figure~\ref{fig:lansing_intensity_separate} that 
the intensity profiles exhibit significant interspecies variability. For exampe
whiteoak seems to have almost homogenous intensity whereas maple
displays a much more inhomogenous pattern. Gaussian kernel smoothed
intensity estimates, separately estimated for each species, is displayed in  
figure~\ref{fig:lansing_intensity_separate}. Another Gaussian kernel smoothed
 intensity estimate is displayed in figure~\ref{fig:lansing_intensity_relative},
 but in this version the intensities are directly comparable between the species.



\section{Methods}

\section{Results}

\section{Conclusion}


\printbibliography
\appendix
\section{R code}

%\lstinputlisting[caption={m-code in exercise 1},linerange=1-132]{ex1.m}
%\lstinputlisting[caption={m-code in exercise 2},linerange=1-81]{ex2.m}

\end{document}
