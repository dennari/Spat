\documentclass[12pt,a4paper,oneside,article]{memoir}

% LAYOUT
\setulmarginsandblock{0.7\uppermargin}{0.8\lowermargin}{*}
\usepackage{mylayout}


% LANGUAGE AND FONT
\setmainlanguage{english}
\setmainfont[Ligatures={TeX},Mapping={tex-text},Numbers={OldStyle}]{Linux Libertine O}
\setmonofont[Scale=0.8]{DejaVu Sans Mono}

% PDF SETUP
\usepackage[unicode,bookmarks, colorlinks, breaklinks,
pdftitle={S-114.4202: Exercise report},
pdfauthor={Ville Väänänen},
pdfproducer={xetex}
]{hyperref}
\hypersetup{linkcolor=black,citecolor=black,filecolor=black,urlcolor=MidnightBlue} 
\usepackage{mcode}

\usepackage[shadow]{todonotes}

% MATH
\usepackage{mymath}
\everymath{\displaystyle}


\newcommand{\Th}{\gv{\theta}}
\newcommand{\LH}{\Pdf{\v{Y}}{\Th}}
\newcommand{\LHf}[1]{\Pdf{\v{Y}}{#1}}
\newcommand{\LHh}{\Pdf{\v{Y}}{\hat{\Th}}}
\newcommand{\lLH}{L\!\left(\Th\right)}
\newcommand{\cLH}{\Pdf{\v{X},\v{Y}}{\Th}}
\newcommand{\lcLH}{\log\cLH}
\newcommand{\LB}{\mathcal{L}}
\newcommand{\Lb}{\mathfrak{L}}
\newcommand{\KL}[2]{\mathrm{KL}\left(#1\|#2\right)}


\newcommand{\course}{S-114.4202}
\newcommand{\coursename}{Special Course in Computational Engineering II}
\newcommand{\duedate}{\today}
\newcommand{\studentid}{63527M}
\renewcommand{\title}{Bear and wolf populations in Finland}
\author{Ville Väänänen}

% SECTION NUMBERING
%\setsecnumdepth{subsubsection}
\counterwithout{section}{chapter}
%\renewcommand{\thesubsubsection}{\thesubsection{\large\scshape\alph{subsubsection}}}
%\titleformat{\subsubsection}{\large\scshape}{\alph{subsubsection} )}{10pt}{}
%\titleformat{\section}{\Huge}{\thesection}{10pt}{}
%\titleformat{\subsection}{\Large}{\thesubsection}{10pt}{}


% BIBLIOGRAPHY
\usepackage[hyperref=true,backend=biber]{biblatex} % TEXLIPSE BUG: backend cannot be first
%\addbibresource{SSM_PE.bib}


\checkandfixthelayout

\begin{document}


\mytitlepage

\section{Data description}

The bear and wolf observations for the year 2009 are plotted in figure~\ref{fig:bw_2009}.
As can be seen, there is some clustering which has to be associated with increased
probability of reporting due population density. Thus an interesting problem is,
how should we take into account the probability for an animal to get observed
in a given area? Is it directly proportional to population density? Is there some
other factors associated such as the amount of hunters in the area?

\begin{figure}[htbp]
  \begin{adjustwidth}{-2in}{-2in}
	  \centering
	  \subfloat[Bear]{\label{fig:bear}\includegraphics[width=0.6\textwidth]{bear_2009}}
	  \subfloat[Wolf]{\label{fig:wolf}\includegraphics[width=0.6\textwidth]{wolf_2009}}
  \end{adjustwidth}
  \caption{Bear and wolf observations in 2009}
  \label{fig:bw_2009}
\end{figure}

\section{Methods}

\section{Results}

\section{Conclusion}

\end{document}
